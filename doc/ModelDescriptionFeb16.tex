
\documentclass{report}
%%%%%%%%%%%%%%%%%%%%%%%%%%%%%%%%%%%%%%%%%%%%%%%%%%%%%%%%%%%%%%%%%%%%%%%%%%%%%%%%%%%%%%%%%%%%%%%%%%%%%%%%%%%%%%%%%%%%%%%%%%%%%%%%%%%%%%%%%%%%%%%%%%%%%%%%%%%%%%%%%%%%%%%%%%%%%%%%%%%%%%%%%%%%%%%%%%%%%%%%%%%%%%%%%%%%%%%%%%%%%%%%%%%%%%%%%%%%%%%%%%%%%%%%%%%%
\newtheorem{theorem}{Theorem}
\newtheorem{acknowledgement}[theorem]{Acknowledgement}
\newtheorem{algorithm}[theorem]{Algorithm}
\newtheorem{axiom}[theorem]{Axiom}
\newtheorem{case}[theorem]{Case}
\newtheorem{claim}[theorem]{Claim}
\newtheorem{conclusion}[theorem]{Conclusion}
\newtheorem{condition}[theorem]{Condition}
\newtheorem{conjecture}[theorem]{Conjecture}
\newtheorem{corollary}[theorem]{Corollary}
\newtheorem{criterion}[theorem]{Criterion}
\newtheorem{definition}[theorem]{Definition}
\newtheorem{example}[theorem]{Example}
\newtheorem{exercise}[theorem]{Exercise}
\newtheorem{lemma}[theorem]{Lemma}
\newtheorem{notation}[theorem]{Notation}
\newtheorem{problem}[theorem]{Problem}
\newtheorem{proposition}[theorem]{Proposition}
\newtheorem{remark}[theorem]{Remark}
\newtheorem{solution}[theorem]{Solution}
\newtheorem{summary}[theorem]{Summary}
\newenvironment{proof}[1][Proof]{\noindent\textbf{#1.} }{\ \rule{0.5em}{0.5em}}
\setlength{\parskip}{1.8mm}
\setlength{\parindent}{0mm}
\begin{document}

\title{Housing ABM}
\maketitle
\tableofcontents
\chapter{Model Description}

\section{Introduction}
The model consists of houses that are bought and sold by households on a housing market; households take out mortgages from a bank which is regulated by a central bank. Owner-occupiers can also choose to buy and sell properties as buy-to-let investments. These are offered on a rental market and rented by households who decide not to buy. Households that cannot afford to rent or buy are put into social housing.

Houses have no intrinsic properties other than a single `quality', which acts as a proxy for size, location, condition etc. Quality bands are assigned so that there are roughly the same number of houses in each band; at present there are 48 bands. The model is time-stepping with a step of one month.

A note on equations: the meaning of each variable in an equation will be given directly after the equation. If two equations use the same symbol and the symbols have differing definitions, then there is no implied connection between the two occurrences of the symbol.

\section{Household lifecycle}
\label{lifecycle}
Households enter the model, age, and exit. The total `birth' rate of households is held constant\footnote{Based on a target population of 10,000 households and a per capita birth rate of 0.0102 per year, calculated against the flux of first time buyers (CML, 2015)}. Upon birth, households are endowed with an age drawn from a beta distribution in order to give a finite support and to ensure a minimum age for leaving home\footnote{The distribution has $\alpha$ and $\beta$ equal to 2, shifted and scaled to have a support from 14.5 to 44.5 years (ONS, 2009) This does not account for new households created by divorce/separation.}, a value representing their income percentile (drawn from a uniform distribution) and wealth\footnote{Given by $\omega$ in equation \ref{incomewealth}} but no existing housing. 8\% of households that are above the 50th percentile of income are given a buy-to-let `gene' which gives them the desire to enter the buy-to-let market\footnote{Combining number of private rental houses (ONS, 2011) with the distribution of number of houses owned by BTL investors (ARLA, 2014) gives 4\% of population are BTL investors. We assume that BTL investors are more likely to be in the higher income bracket, this is not calibrated.}.

The age of a household represents the age of the `household reference person' (HRP) (a concept that exists in many household surveys). As a household ages, its income percentile remains fixed but its actual income changes to reflect the income distribution over households of that age (ONS, 2013)\footnote{The STATA script to extract the distribution from the raw data, and the subsequent analysis can be found in the repo at \texttt{calibration/IncomeDistribution.xlsx}.}. Income is bounded by a lower limit of \pounds 5,900 (the current level of income support for a married couple).

\subsection{Death and Inheritance}
Each month, each household has a probability of `death' given by
\begin{equation}
P_{die} = Ae^{ka}
\label{mortality}
\end{equation}
where $a$ is age and $k$ and $A$ are constants\footnote{$A=7.576e-9$ and $k=0.148$. This ensures that the oldest households are around 100 years old and the population is 10,000 for the given total birth rate.}. This exponential mortality rate is seen in data for individuals. In the case of households, marriage is also a cause of death; this is not accounted for in the model.

On exit, all of a household's financial and housing wealth is given to another, randomly chosen, household. If the deceased household had any houses on any market, they are taken off the market. Any tenants living in the houses are evicted. If the household was renting, the rental contract is terminated (this isn't realistic). Outstanding mortgages are written off (this isn't realistic).

Upon inheriting a house, renters and the socially housed will immediately move into it, an owner-occupier will immediately sell it and a buy-to-let investor will decide whether to sell it or add it to their portfolio according to the decision rule in section \ref{sellbtl}.

The lifecycle was primarily included to provide a flux of first time buyers, who behave differently and are treated differently by banks, but including it allows many other dynamics to be investigated.
 
\section{Simulation initialisation}
In order to initialise a simulation with a realistic assignment of houses and mortgages to households, the model goes through a `spin-up' period before any simulation. The spin-up period begins with no households and no houses. Households are born, age and die as described in section \ref{lifecycle}. Under these conditions, the model population will naturally grow until the total death rate (given by the integral over age of mortality rate times population) equals the total birth rate (which is held constant).

During spin-up, houses are put on the sale market by a `construction sector' whenever the household to house ratio falls below a fixed value\footnote{household to privately-owned house ratio is 1.22 (ONS, 2011)}. New houses will be put onto the house sale market at a price based on the ONS house price index data tables for 2013. If unsold, the price will be reduced at a rate of $5\%$ per month.

We typically allow the model to spin-up for around 200 years.

\section{Households}

\subsection{A month in the life...}
In each one-month time-step, each household:
\begin{enumerate}
\item ages by 1 month and possibly dies, leaving an inheritance
\item receives its gross employment income and pays income tax and national insurance according to UK tax law for a single person in the 2014/15 tax year (this needs changing to account for probability of household status given income and age, to account for married couples and multiple adult occupancy).
\item makes mortgage and/or rental payments and collects any rent due
\item consumes
\item if in social housing or at end of tenancy decides whether to try to rent/buy a new house
\item if a buy-to-let investor decides whether to buy more investment properties
\item decides whether to sell any owned houses
\item rethinks the offer price of any house currently on the rental/sale market or takes the house off the market.
\end{enumerate}


\subsection{Expected House-price growth}
\label{housepricegrowth}
A household's expectation for annual house price growth, $\bar{g}$, is equal to the last year's growth in the quarterly HPI. So
\begin{equation}
\bar{g} = \alpha\left(\frac{h_0 + h_{1} + h_{2}}{h_{12} + h_{13} + h_{14}} - 1\right)
\end{equation}
where $\alpha$ is a constant parameter and $h_t$ is the monthly house price index $t$ months ago.

\subsection{Household consumption}

Households have a fixed, subsistence-consumption set at the married couple's monthly lower earnings limit for UK income support (\pounds 5,900). After this is subtracted from income, the household's discretionary consumption is calculated in the following way: Suppose the household has a `minimum acceptable liquid wealth' defined as
\begin{equation}
\ln(\omega) = 4.07\ln (i)-33.1+\varepsilon
\label{incomewealth}
\end{equation}
where $i$ is gross annual income and $\varepsilon$ is a Gaussian noise term. Now define household consumption over the month as
\begin{equation}
E=0.5 \max \left( b-\omega,0\right)
\label{consumption}
\end{equation}
where $b$ is the household's liquid wealth (after receiving this month's employment and rental income, paying tax, rent, mortgage and subsistence-consumption).

This formula ensures that the aggregate (liquid) wealth distribution fits ONS (2012), while ensuring that households with higher income consume more. Since both income and wealth distributions are approximately log-normal (ONS, 2012 and Belfield et.al. 2014), equation \ref{incomewealth} can be understood as a transformation from a log-normal income distribution to a log-normal wealth distribution.

This consumption equation has the effect of making the actual wealth of a household relax exponentially towards a value just above the minimum acceptable value, $\omega$. The rate of this relaxation is quite aggressive, effectively making actual wealth a noisy function of income. This needs to be improved.

\subsection{Decisions while in ``social housing''}
All agents are born into ``social housing''. Although we refer to this as social housing, this also represents homelessness, living with parents while looking for a house or living in temporary accommodation (e.g.hotel, staying with friends) while between houses.

Agents never choose to be in ``social housing'', but are put there if they fail to secure any other form of housing at a given time. If they find themselves in social housing they will always consider renting or buying and will bid on the appropriate market. When in social housing, no rental payments are deducted from income, this is a very simple form of housing benefit.

\subsubsection{Decision to rent or buy a home}
\label{rentorbuy}
If an agent is in need of a new home (if in social housing, at the end of a rental contract or directly after the sale of a house), they need to decide between renting and buying. To do this they first decide on a price they would pay to buy a house, $p$, calculated as the minimum of the desired house price according to section \ref{buyahome} and the maximum mortgage the bank is willing to finance. They then check the current market prices to see what quality of house they can expect to get for this price. The probability of deciding to buy is then given by
\begin{equation}
P_{buy} = \frac{1}{1 + e^{-K_{rb}(C_{r}(1+C_R) - (m - \bar{g}p))}}
\end{equation}
where $K_{rb}$ is a constant giving sensitivity to cost, $C_{r}$ is the current market annual rent on a house of the same quality they would expect to buy, $C_R$ is a constant, representing the intrinsic desire to own rather than rent (i.e. the psychological cost of renting), $m$ is the expected annual mortgage payment and $\bar{g}$ is the expected annual house price growth as defined in section \ref{housepricegrowth}.

\subsection{Decisions as a renter}
If a household decides to rent, they will bid 0.3 times their income for rent\footnote{An empirical pdf of income against rent was derived from Zoopla data but I haven't got around to putting this into the model. The pdf is in the Github repo at \texttt{code/pdfRentalPrice.hd5} and the script used to create it is at \texttt{code/RentalPrice.py}}. Upon entering a rental contract, they will live in the rented house and pay rent until the end of the contract (unless evicted by the executors of the will of a deceased landlord). At the end of the contract, they will reconsider whether to rent or buy as described in section \ref{rentorbuy} and re-enter the appropriate market.

\subsection{Decisions as a Homeowner}

\subsubsection{Bidding for a home}
\label{buyahome}
If a household decides to buy a home, it will bid on the house sale market. The desired amount of the bid is given by
\begin{equation}
p = \frac{\sigma i e^{N(0,\epsilon)}}{1 - A\bar{g}}
\end{equation}
where $i$ is income, $\bar{g}$ is expected house price growth, $N()$ is Gaussian noise, $\sigma$, $A$ and $\epsilon$ are parameters. This formula is exactly equivalent to the one used in Axtell et.al. (2014) with different parameter values.

We can make one possible intuitive interpretation of this equation by multiplying left and right by $Bt-BtA\bar{g}$ and noting that $pBt$ represents the transaction costs of buying the house plus the maintenance costs and mortgage interest payments over a period $t$, $pt\bar{g}$ is the expected capital appreciation of the house over time $t$ and $AB$ represents the household's rate of conversion between illiquid housing wealth and liquid wealth. The household is then choosing to bid an amount such that the overall cost of the house over the period $t$ is some (noisy) fraction, $B\sigma$, of their income over the same period. This equation could be improved by adding equations for transaction and maintenance costs, calculating mortgage interest payments from interest rate and LTV and integrating over some time horizon, possibly with future discounting.

The actual amount bid is the closest amount possible to the desired bid, after accounting for any bank-decided constraints on mortgages available to the agent.

\subsubsection{Downpayment on a new home}
\label{downpayment}
On buying a house, the minimum downpayment on the house is imposed by the mortgage lender during the mortgage pre-approval process (where applicable) but the household may choose to make a larger downpayment. If the household has liquid wealth of 1.25 times the price of the house, they will pay outright (this ensures there are roughly the right number of cash buyers). Otherwise they will choose the $i^{th}$ percentile from a log-normal distribution calibrated against emprical distributions of LTV among newly issued mortgages (Bank of England, 2015), where $i$ is their income percentile. The parameters of the log-normal distributions are different for first time buyers (FTB) and owner-occupiers (OO) in line with data in Bank of England (2015).

\subsubsection{Decision to sell a home}
The probability that an agent will sell their home is given by
\begin{equation}
p = c(1 + a(0.05-n_h) + b(0.03-i))
\label{sellhome}
\end{equation}
where $a$, $b$ and $c$ are constants, $n_h$ is the number of houses per capita currently on the market and $i$ is the mortgage interest rate (expressed as percent/100)\footnote{$c=0.007575$, $a=4$ and $b=5$. a and b are fudge factors. c is calibrated against average house sales, British housing survey 2008.}. This is a fudge to prevent unrealistic build up of housing stock on the market and unrealistic fluctuations in interest rates.

While a household's house is for-sale, they will not attempt to look for another home. Once their house is sold they will be made temporarily homeless and only then will they bid for a new home. This needs to be addressed.

\subsubsection{Sale price decision}
\label{saleprice}
Houses are offered on the market at a price, $q,$ given by

\begin{equation}
\ln q=C+\ln (\bar{p})-D\ln \left( \frac{\bar{d}+1}{31}\right) +\varepsilon
\label{salepriceeq}
\end{equation}

where $C=0.095$, $\bar{p}$ is the average sold-price of houses of this quality, $\bar{d}$ is the average days on the market for all house qualities, $D$ is a tunable parameter (currently set to 0), and $\varepsilon=N(0,0.01^{2})$. Please see Axtell et.al. (2014) in the Github repo for motivation for this equation.

If a house remains on the market from the previous time-step, with a 6\%
probability its price is reduced. Records of price reductions on Zoopla showed a very good fit to a Gaussian in the log domain, so the model reduces according to
\begin{equation}
p' = p\left(1-e^{N(\mu,\sigma)}\right)
\label{reprice}
\end{equation}
where $N(\mu,\sigma)$ is a draw from a Gaussian distribution.  \footnote{Calibrated against Zoopla data. The Python code used to extract the Gaussian fit can be found in the repo under \texttt{calibration/code/SaleReprice.py}, and the output of the script is at \texttt{calibration/HouseRepriceAnalysis.html}}. If the price drops below the amount needed to pay the mortgage on the house, it is withdrawn from the market.

\subsection{Buy-To-Let Investor's decisions}
\subsubsection{BtL heterogeneity}
BtL investors are, with a tunable probability, randomly assigned to be either `fundamentalist' or `trend follower'. The only difference between the two is the value of the `capital gain coefficient' $c_{g}$, which is used in some of the decision processes below.

\subsubsection{Buy-to-let rental offers}
A BTL investor will put a house on the rental market whenever a rental contract ends, or when a new buy-to-let house is bought or inherited that doesn't already have a tenant.

BTL investors offer houses on the rental market at a monthly rental given by:

\begin{equation}
\ln q=C+\ln (\bar{p})-D\ln \left( \frac{\bar{d}+1}{31}\right) +\varepsilon
\end{equation}

where $C=0.01$, $\bar{p}$ is the average monthly rent of houses of this quality, $\bar{d}$ is the average days on the market for all house qualities, $D=0.02$ is a tunable parameter, and $\varepsilon=N(0,0.05^{2})$. This is of the same form as used for the sale price in equation \ref{salepriceeq}.

If a house on the rental market does not get filled, the price is multiplied by 0.95 each month.

The length of a rental agreement is chosen randomly from 12 to 24 months with uniform probability (ARLA, 2014).

\subsubsection{Decision to sell BTL property}
\label{sellbtl}
Buy-to-let investors will consider selling their investment properties at the end of each tenancy agreement, and will re-consider each month until another tenant moves in. The decision to sell is based on the `effective yield' on the house, which is defined as
\begin{equation}
y_e = \frac{2(c_g \bar{g}p + (1-c_g)r) - m}{e} 
\end{equation}
where $c_g$ is the investor's capital gain coefficient $\bar{g}$ is expected annual house price appreciation, $r$ is current annual rental income from the house, $m$ is the current annual mortgage payment and $e$ is the maximum of the current equity in the house and 1 pence.

The probability of deciding to keep the house is then given by
\begin{equation}
P(keep) = \frac{1}{(1 + e^{ky_e+c})^\gamma}
\end{equation}
where $k$ is a scaling constant, $c$ represents transaction costs and stickiness, and $\gamma$ deals with the fact that the decision to keep or sell is made every month in the model, whereas in reality this decision may be made less frequently.

If an investor decides to sell, the house will be taken off the rental market and put on the sale market at the price given in section \ref{saleprice}.

This monthly re-considering means that as a house remains unoccupied on the rental-market it becomes more and more likely to be sold. This is desirable but we may want to consider more carefully the time evolution of an investor's propensity to sell.

\subsubsection{Decision to buy BTL property}
Buy-to-let investors decide to add houses to their current portfolio based on the `expected yield' on property investments, which is defined as
\begin{equation}
\bar{y} = 2l(c_g \bar{g} + (1-c_g)\bar{r}) - \frac{m}{d} 
\end{equation}
where $c_g$ is the investor's capital gain coefficient, $\bar{g}$ is expected annual house price appreciation, $\bar{r}$ is an exponential average of the gross annual rental yield (rent/house price) on newly issued rental contracts, $l$ is the leverage (house price over downpayment) of the largest mortgage available to the investor, $m$ is the associated annual mortgage payment and $d$ is the minimum downpayment.

The probability of deciding not to buy any house is then given by
\begin{equation}
P(\overline{buy}) = \frac{1}{(1 + e^{k\bar{y}+c})^\gamma}
\end{equation}
where $k$ is a scaling constant, $c$ represents transaction costs and stickiness, and $\gamma$ deals with the fact that the decision to buy or not is made every month in the model, whereas in reality this decision may be made less frequently.

Upon buying an investment property, the property is immediately put onto the rental market.

\subsubsection{Downpayment}
The decision on how much downpayment to make on a newly purchased house is made in the same way as described in section \ref{downpayment} except that BTL investors will choose downpayment from a Gaussian distribution (bounded at the lower end by zero) rather than a log-normal, this is calibrated against confidential BoE data\footnote{I was never told where this data came from}.

\subsection{Bankruptcy}
If a household's liquid wealth goes negative, they are given a cash injection to raise their liquid wealth to 1 pound. Households make no directed attempt to avoid bankruptcy and will not decide to sell housing wealth in response to dwindling liquid wealth.

\section{Banks}

There is a single bank in the model which represents the mortgage lending
sector in the aggregate.

\subsection{Mortgage approval}
\subsubsection{Loans to owner occupiers}
All loans to owner-occupiers are 25 year, fixed interest repayment mortgages. The bank will approve a mortgage to a (potential) owner-occupier as long as it conforms to LTV, LTI and affordability constraints. The affordability constraint ensures that a household has enough total income to pay all its mortgages. Subject to meeting those criteria, all demand is met in any period. The maximum principal loan amount, then, is calculated as

\begin{equation}
q=\min \left( \frac{b\theta}{1-\theta},I \psi ,i\phi\frac{%
1-(1+r_{stress})^{-N}}{r_{stress}}\right)
\end{equation}

The constraints are described in the following table.

\begin{tabular}{p{1in}|p{3.5in}}
Constraint & Description \\ \hline\hline
$\frac{b\theta}{1-\theta}$ & LTV constraint. $b$ is the household's bank
balance (assuming all bank balance is used as downpayment), $\theta$ is the maximum loan to value ratio. \\ 
$I\psi $ & LTI constraint. $I$ is household gross income and $\psi $ is
the maximum loan to income ratio \\ 
$i\phi\frac{1-(1+r_{stress})^{-N}}{r_{stress}}$ & Affordability test
given a monthly payment equal to
the share $\phi=0.5$ of the household's disposable income available for
mortgage payments. $i$ is a household's disposable income, $r_{stress}$ is the fixed monthly interest rate based on
a stress scenario and $N$ is the number of monthly payments to pay off the
mortgage.
\end{tabular}

In the case of LTI and LTV constraints, the central bank allows a certain proportion of loans to owner-occupiers to be unregulated. The bank has its own, higher LTI and LTV ratio limits which it applies to unregulated loans, and it will apply these limits to a loan whenever the approval of the loan would not cause the bank to exceed the maximum proportion set by the central bank.

\subsubsection{Loans to buy-to-let investors}
Mortgages to buy to let investors are interest only, fixed interest. There are no LTI or affordability constraints. Instead, there is a central-bank imposed Interest Coverage Ratio limit. The ICR limit imposes the constraint that
\begin{equation}
q < \frac{b}{1 - \frac{\bar{y}}{\xi I}}
\end{equation}
where $\bar{y}$ is the exponential average of the gross annual rental yield on new tenancy agreements (i.e. gross annual rental income over house price), $\xi$ is the ICR and $I$ is a stressed mortgage interest rate of 5\%.

Central bank regulation of LTV on loans to BTL investors is configurable. At present there is no central bank regulation of LTV for BTL investors, so the bank sets its own LTV limit.

\subsection{Interest rates}
Mortgage interest rate spread, $r$, is calculated each month according to
\begin{equation}
r_{t+1} = r_{t} + k(S_t-T)
\label{spread}
\end{equation}
where $k$ is a constant, $S_t$ is the total supply of credit in month $t$ and $T$ is an exogenous constant target monthly supply\footnote{calibrated against BoE historic data on credit supply and interest rate spread, please ask Marc Hinterschweiger for more details.}.

\section{Housing markets}

\subsection{Market price information}

Households make use of market price information when making their decisions. For both the rental and sale markets, two types of information are available: the house price index and the market price of a house of a given quality.

The house price index for a given month is defined as the average transaction price divided by the average reference price over the set of all completed transactions for that month. The reference price of a house is the price of a house of that quality according to the ONS house price data tables 2013. This is an approximation to Nationwide's mix adjustment methodology. Since the houses in the model have only one property, namely quality, the hedonic regression can be reduced to a regression on $p = \chi p_{r}$ where $p$ is the price of a house now, $\chi$ is the house price index and $P_{r}$ is the reference price, which can be thought of as a measure of quality.

The market price given quality is calculated as a moving exponential average of completed transactions involving houses of the given quality. Because the number of transactions per month may be quite small in the simulation (due to scaling down of the population) some quality bands may have very few transactions which leads to unrealistic distributions of price with quality. Analysis of house price distribution data over time shows that the shape of the distribution stays the same (almost log-normal). When the simulated population is large the model naturally reproduces this.

To deal with scaled-down populations, however, at the end of every month the market price given quality is transformed according to
\[
p_q' = Dp_q + (1-D)hp_r(q)
\]
where $D$ is a constant, $p_q$ is the market price of quality $q$, $h$ is the house price index for this month and $p_r(q)$ is the reference price of houses of that quality. This effectively relaxes the distribution of house prices to the shape (but not the level) of that in the 2013 data tables. This is a quick and dirty fix. Initial experiments using just HPI, and using a regression of the form $p_q = \chi p_{qr} + c$, led to unrealistically unstable prices. BoE has historical data on real house-price distributions which could be used to back out a low-dimensional family of distribution functions which could then be used as the basis for a regression.

\subsection{House sales clearing}

Clearing proceeds as follows: Home-buyers are matched to the best quality
house they can afford and BtL investors are matched to the best expected-gross-rental-yield house they can afford. The expected-gross-rental-yield of a house of quality $q$, for-sale at a price $p$ is defined as
\begin{equation}
E(y_q) = \frac{12\overline{r_q}E(o_q)}{p}
\end{equation}
where $\overline{r_q}$ is the current market monthly rental price of houses of quality $q$ and $E(o_q)$ is the expected occupancy of a rental property of quality $q$, (i.e. the expected fraction of time that rent will be collected on the property); this is based on an 18 month rental contract followed by a number of days waiting for the next tenant, giving
\begin{equation}
E(o_q) = \frac{547}{547+\overline{D_q}}
\end{equation}
where $\overline{D_q}$ is the exponential average of the number of days that newly rented properties of quality $q$ spent on the rental market.

When a given offered house is matched with more than one bidder, the price is `bid up' by multiplying by $1.0075^k$ where $k$ is chosen at random from a geometric distribution such that
\begin{equation}
p(k) = (1-e^{-7b/30})^{k-1}e^{-7b/30}
\end{equation}
where $b$ is the number of bids received in the timestep. The house is then offered to a randomly chosen bid that can still afford to buy. This approximates the outcome that would be achieved if the bids came in on random days in the simulated month; if a bid is followed by another bid within 7 days, the new bid `bids up' the price by $0.75\%$, the first bid that is not bid up within 7 days gets the house.

Failed bids then get to bid again. This re-bidding carries on up to the smaller of $N/1000$ and $1+n/5000000$ times, where $N$ is the population and $n$ is the total number of orders on the market.

\subsection{Rental clearing}

Rental clearing proceeds in the same way as house-sales-clearing, but
without yield-driven BtL bidders.\bigskip

\section{Central bank}

The central bank sets LTV, LTI, interest cover ratio (ICR) and affordability
policies. Policies can be of three different types:

1. Strict limits, e.g. a hard LTV limit of 90\% for all households (though
the limit may differ between types of agents, such as first-time buyers or
owner-occupiers);

2. 'Soft' limits, e.g. an LTI cap of 3.5 on new mortgage lending, but
allowing for 15\% of new mortgages above this limit;

3. State-contingent policies, e.g. an LTV limit of 85\% if credit growth
over a certain time is above a certain threshold; otherwise no limit.


\appendix
\begin{thebibliography}{20}

\bibitem{ARLA:14} ARLA, \textbf{2014}: The ARLA Review and Index for Residential Investment, second quarter 2014. Association of Residential Letting Agents, Warwick.

\bibitem{AxtellEA:14} Axtell et.al., \textbf{2014}: An Agent-Based Model of the Housing Market Bubble in Metropolitan Washington, D.C. available on the \texttt{housingPrivate} Github repo at \texttt{doc/AxtellEA14.pdf}.

\bibitem{PSD:15} Bank of England, \textbf{2015}: Product Sales Database, bank confidential.

\bibitem{BelfieldEA:14} Belfield, C. et.al. \textbf{2014}: Living Standards, Poverty and Inequality in the UK: 2014, Institute of Fiscal Studies, London.

\bibitem{CML:15} CML, \textbf{2015}: Regulated Mortgage Survey, Council of Mortgage Lenders

\bibitem{ONS:09} ONS, \textbf{2009}: The changing living arrangements of young adults in the UK, Population Trends, winter 2009, Office for National Statistics, Newport.

\bibitem{ONS:11} ONS, \textbf{2011}: Social Trends: Housing 2011, Office for National Statistics, Newport.

\bibitem{ONS:12} ONS, \textbf{2012}: Wealth in Great Britain 2010/2012, table 5.8, Office for National Statistics, Newport.

\bibitem{ONS:13} ONS, \textbf{2013}, Living Costs and Food Survey, Office for National Statistics, Newport.

\bibitem{ONS:14} ONS, \textbf{2014}: House Price Index Reference Tables - annual tables 20 to 39, table 34, Office for National Statistics, Newport.

\end{thebibliography}

\end{document}
