
\documentclass{article}
%%%%%%%%%%%%%%%%%%%%%%%%%%%%%%%%%%%%%%%%%%%%%%%%%%%%%%%%%%%%%%%%%%%%%%%%%%%%%%%%%%%%%%%%%%%%%%%%%%%%%%%%%%%%%%%%%%%%%%%%%%%%%%%%%%%%%%%%%%%%%%%%%%%%%%%%%%%%%%%%%%%%%%%%%%%%%%%%%%%%%%%%%%%%%%%%%%%%%%%%%%%%%%%%%%%%%%%%%%%%%%%%%%%%%%%%%%%%%%%%%%%%%%%%%%%%
\newtheorem{theorem}{Theorem}
\newtheorem{acknowledgement}[theorem]{Acknowledgement}
\newtheorem{algorithm}[theorem]{Algorithm}
\newtheorem{axiom}[theorem]{Axiom}
\newtheorem{case}[theorem]{Case}
\newtheorem{claim}[theorem]{Claim}
\newtheorem{conclusion}[theorem]{Conclusion}
\newtheorem{condition}[theorem]{Condition}
\newtheorem{conjecture}[theorem]{Conjecture}
\newtheorem{corollary}[theorem]{Corollary}
\newtheorem{criterion}[theorem]{Criterion}
\newtheorem{definition}[theorem]{Definition}
\newtheorem{example}[theorem]{Example}
\newtheorem{exercise}[theorem]{Exercise}
\newtheorem{lemma}[theorem]{Lemma}
\newtheorem{notation}[theorem]{Notation}
\newtheorem{problem}[theorem]{Problem}
\newtheorem{proposition}[theorem]{Proposition}
\newtheorem{remark}[theorem]{Remark}
\newtheorem{solution}[theorem]{Solution}
\newtheorem{summary}[theorem]{Summary}
\newenvironment{proof}[1][Proof]{\noindent\textbf{#1.} }{\ \rule{0.5em}{0.5em}}
\setlength{\parskip}{1.8mm}
\setlength{\parindent}{0mm}
\begin{document}

\title{Housing ABM, December 2015}
\maketitle

\section{Introduction}
The model consists of a construction sector that builds houses that are bought and sold by households on a housing market. Households take out mortgages from a bank, which is regulated by a central bank. Owner-occupiers can also choose to buy and sell properties as buy-to-let investments. These are offered on a rental market and rented out by households that decide to rent rather than buy. Households that cannot afford to rent or buy are put into social housing.

Houses have no intrinsic properties other than a single `quality', which acts as a proxy for size, location, condition etc. The model is time-stepping with a step of one month.

\section{Household lifecycle}

Households enter the model, age, and exit. Upon entry, households are assigned an income percentile, which they will retain for the rest of their lives. They are endowed with wealth, based on average wealth for their income, but no existing housing. 8\% of households that are above the 50th percentile of income are given a buy-to-let `gene' which gives them the desire to enter the buy-to-let market.

The age of a household represents the age of the `household reference person' (HRP) (a concept that exists in many household surveys). As a household ages, its income changes based on its income percentile and the aggregate income distribution for that age. Income has a lower bound of \pounds 5,900 (the current level of income support for a married couple).

On exit, all of a household's financial and housing wealth is given to another, randomly chosen, household.
 
\section{Simulation initialisation}
The model goes through a `spinup' peroid before any simulation. The spinup peroid begins with no households and ends with 10,000 households. Spinup can proceed in one of two modes. If the simulation requires a non-equilibrium population, the birth rates during spinup are calculated so that at the end of the spinup, the households have the desired age distribution. If we want to simulate an equilibrium age distribution for a given birth and deeath rate (the current default), birth and death rates are held constant during spinup so that population hits the equilibrium distribution after around 100 years.

\section{Households}

In each period, each household receives its gross employment income and any income from rental properties. Income tax is paid according to national tax rates to leave disposable income. Mortgage holders make mortgage payments under the assumption of a 25 year repayment mortgage (changes in interest rates do not affect existing owners). Renters pay their rent (same as last period unless just moved).

\subsection{Household consumption}

Households consume according to

\begin{equation}
E=\alpha \max \left( b-e^{4.07\log (i)-33.1+\varepsilon },0\right)
\end{equation}

where $b$ is the household's liquid wealth and $i$ is gross annual income.

\subsection{Decisions while in ``social housing''}
All agents are born into ``social housing''. Although we refer to this as social housing, this also represents homelessness, living with parents while looking for a house or living in temporary accommodation (e.g.hotel, staying with friends) while between houses.

Agents never choose to be in ``social housing'', but are put there if they fail to secure any other form of housing at a given time. If they find themselves in social housing they will always consider renting or buying and will bid on the appropriate market.

\subsubsection{Decision to rent or buy a home}
If an agent is in need of a new home (if in social housing, at the end of a rental contract or directly after the sale of a house), they need to decide between renting and buying. The probability of deciding to buy is given by
\[
P_{buy} = \frac{1}{1 + e^{K_{rb}(C_{r}(1+C_R) - C_{b})}}
\]
where $K_{rb}$ is a constant, $C_{r}$ is the average annual rent on a house of the buyer's desired quality, $C_R$ is a constant, representing the intrinsic cost of renting rather than owning, and $C_{b}$ is the expected annual mortgage payment minus the expected annual capital appreciation of a house of the buyer's desired quality.

Desired quality is the maximum quality house that, at current market prices, the household would expect to get if they entered the house sale market (see section \ref{buyahome}).

\subsection{Decisions as a Homeowner}

\subsubsection{Buying a home}
\label{buyahome}
If a household decides to buy a home, it will bid on the house sale market. The desired amount of the bid is given by
\[
 \frac{\sigma i e^{N(0,\epsilon)}}{1.0 - AP}
\]
where $i$ is income, $P$ is expected house price appreciation, $N()$ is Gaussian noise, $\sigma$, $A$ and $\epsilon$ are parameters.

The actual amount bid is the closest amount possible to the desired bid, after accounting for any bank-decided constraints on mortgages available to the agent.

\subsubsection{Downpayment on a new home}
On buying a home, if the household has liquid wealth of 1.25 times the price of the house, they will pay outright. Otherwise they will choose the $i^{th}$ percentile from a log-normal distribution (parameters depending on whether they are FTB or OO), where $i$ is their income percentile (calibrated against emprical LTV distributions).

\subsubsection{Decision to sell a home}
The probability that an agent will sell their home is a linear function of the number of houses currently on the market and the mortgage interest rate.

\subsubsection{Sale price decision}
Houses are offered on the market at a price, $q,$ given by

\begin{equation}
\ln q=0.095+\ln (\bar{p})-D\ln \left( \frac{\bar{d}+1}{\left\langle
d\right\rangle +1}\right) +\varepsilon
\end{equation}

\bigskip

where $\bar{p}$ is the average sold-price of houses of this quality, $\bar{d}
$ is the average days on the market for all house qualities, $\left\langle
d\right\rangle $ is the prior expectation of the number of days on the
market, $D$ is a tunable parameter (currently set to 0), and $\varepsilon
=N(0,0.01^{2})$.

If a house remains on the market from the previous time-step, with a 6\%
probability its price is reduced by an amount drawn from a Gaussian
distribution with mean around -1.6\% and SD 0.6\%. This is calibrated
against data on house price reductions from Zoopla. If the price drops below the amount needed to pay the mortgage on the house, it is withdrawn from the
market.

\subsubsection{Decision on how much to spend on rent}
If a household decides to rent, they will bid 0.3 times their income for rent.

\subsection{Buy-To-Let Investor's decisions}
\subsubsection{Buy-to-let rental offers}
A BTL investor will put a house on the rental market
whenever a rental contract ends, or when a new buy-to-let house is bought
that doesn't already have a tenant. 

The rent BTL investors charge is given by:

\begin{equation}
r=e^{C+log(\bar{p})-D\ast log((d+1.0)/31.0)+E\ast N(0,1)}
\end{equation}

\bigskip

where $C=0.01$, $D=0.02$, $E=0.05$. $N(0,1)$ is a Gaussian noise term with
average 0 and variance 1. $\bar{p}$ is the average mark-to-market rental
price for house of this quality.

If a house on the rental market does not get filled, the price is multiplied by 0.95 each month.

The length of a rental agreement is
chosen randomly from 12 to 24 months with uniform probability. This is based
on figures from ARLA.

\subsubsection{Decision to sell BTL property}

Buy-to-let investors decide to sell houses in their current portfolio based
on the realised interest coverage ratio of that house and the expected
capital gains on the house. BtL investors differ in the weightings they
assign to these two streams of income. The weighted sum of these streams is
then passed through a logistic function to give a probability for deciding
to sell the house.

The price at which the property is offered is decided in the same way as with homes.


\subsubsection{Decision to buy BTL property}

BtL investors decide to buy houses based on the expected yield and expected
capital gain of the best performing house quality on the market. Different
investors put different weightings on these two income streams. The sum of
the weighted streams is passed through a logistic function to get a
probability of bidding on the house-sale market.

The maximum amount they can invest is determined by the LTV, ICR and
affordability constraints that the bank imposes as defined below.

A BTL investor will be willing to invest up to the amount of the maximum mortgage the bank is willing to approve.

\subsubsection{Downpayment}
On buying a house, if the household has liquid wealth of 1.25 times the price of the house, they will pay outright. BTL investors will choose downpayment from a Gaussian distribution, calibrated against data.

\section{Banks}

There is a single bank in the model which represents the mortgage lending
sector in the aggregate.

\subsection{Mortgage approval}
The bank will approve a home mortgage as long as it
conforms to LTV, LTI and affordability constraints. The affordability
constraint ensures that a household has enough total income to pay all its
mortgages. Subject to meeting those criteria, all demand is met in any
period. The maximum mortgage amount, then, is calculated as

\bigskip 
\begin{equation}
q=\min \left( \frac{b(1-\theta )}{\theta },i\ast \psi ,\phi \ast pdi\frac{%
1-(1+r_{stress})^{-N}}{r_{stress}}\right)
\end{equation}

\bigskip

In the case of buy to let investors, a central-bank imposed Interest Coverage Ratio is imposed in place of the income to value ratio.

\bigskip The constraints are described in the following table.

\bigskip

\noindent \bigskip 
\begin{tabular}{p{1.5in}|p{4in}}
Term / constraint & Description \\ \hline\hline
$\frac{b(1-\theta )}{\theta }$ & LTV constraint. $b$ is the household's bank
balance, $\theta $ is the minimum loan to value haircut (i.e. 1 minus the
loan to value ratio). $\theta $ is 0.1 for first-time buyers, 0.2 for
homeowners moving home and 0.4 for buy-to-let investors. \\ 
$i\ast \psi $ & LTI constraint. $i$ is household gross income and $\psi $ is
the maximum loan to income ratio \\ 
$\phi \ast pdi\frac{1-(1+r_{stress})^{-N}}{r_{stress}}$ & Affordability test
given a down-payment of all cash in the bank and a monthly payment equal to
the share $\phi $ of the household's disposable income available for
mortgage payments. $pdi$ is a household's disposable income (including
income from rent), $r_{stress}$ is the fixed monthly interest rate based on
a stress scenario and $N$ is the number of monthly payments to pay off the
mortgage. $\phi $ is 0.25.%
\end{tabular}

\subsection{Interest rates}
Mortgage interest rate spread, $r$, is calculated according to
\[
r_{t+1} = r_{t} + k(S-T)
\]
where $k$ is a constant, $S$ is the aggregate rate of supply of credit and $T$ is a constant target rate of supply.

\section{Housing market clearing}

\subsection{House sales clearing}

Clearing proceeds as follows: Home-buyers are matched to the best quality
house they can afford and BtL investors are matched to the best yield house
they can afford. Where a given offered house is matched with more than one
bidder, the price is `bid-up' and offered at random to one of the bids that
can still afford to buy. For a total number, $b$, of bids received in the timestep, the number of bid-ups, $k$, is drawn from a geometric distribution such that:
\[
p(k) = (1-e^{-7b/30})^{k-1}e^{-7b/30}
\]
Each bid-up increases the price of the house by $0.75\%$.

Failed bids then get to bid again. This re-bidding carries on up to the smaller of $N/1000$ and $1+n/5000000$ times, where $N$ is the population and $n$ is the total number of orders on the market.

\subsection{Rental clearing}

Rental clearing proceeds in the same way as house-sales-clearing, but
without yield-driven BtL bidders.\bigskip

\section{Central bank}

The central bank sets LTV, LTI, interest cover ratio (ICR) and affordability
policies. Policies can be of three different types:

1. Strict limits, e.g. a hard LTV limit of 90\% for all households (though
the limit may differ between types of agents, such as first-time buyers or
owner-occupiers);

2. 'Soft' limits, e.g. an LTI cap of 3.5 on new mortgage lending, but
allowing for 15\% of new mortgages above this limit;

3. State-contingent policies, e.g. an LTV limit of 85\% if credit growth
over a certain time is above a certain threshold; otherwise no limit.

\section{Construction sector}
The construction sector builds houses during the spinup period so as to keep the household to house ratio constant. New houses will be put onto the house sale market at the current market price and will drop the price at a rate of $5\%$ per month if unsold.

\end{document}
