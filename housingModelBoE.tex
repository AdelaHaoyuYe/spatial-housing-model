\documentclass[a4paper]{article}
\usepackage{graphics}
%%\usepackage[english,greek]{babel}
\setlength{\parindent}{0mm}
\setlength{\parskip}{1.8mm}
\newtheorem{definition}{Definition}
\newtheorem{theorem}{Theorem}
\newtheorem{proof}{Proof}
\title{Housing market model}
%%\date{$28^{th}$ August, 2007}
\author{Daniel Tang}
%%\linespread{1.3}

\begin{document}
%%\selectlanguage{english}
\maketitle

%%\tableofcontents

\section{Introduction}

The model consists of $N$ households, $N_h$ houses, a bank, a housing market and a rental market. Households have a fixed monthly income and houses have a fixed "quality" represented as an integer in a given range $[0:Q-1]$. Households can live in social housing, private rented accommodation, be owner-occupiers or buy-to-let investors.

\section{Simulation initialisation}

We create $N=5000$ households and $N_h=4100$ houses. Houses are assigned a current cash value from an empirically tuned log-normal distribution and a quality proportional to the log of the cash value.

\subsection{Households}
Households are given an income from an empirically tuned log-normal distribution. Incomes have a lower bound at the level of income support for a married couple (currently \pounds 5,900). The lowest income households are assigned to social housing (not included in the 4100 houses). The next lowest income households are deemed to be in privately rented accommodation, while the remaining households are deemed to be owner-occupiers (in fact, the probability of being an owner-occupier is a logistic function of income). A percentage (4\%, based on data) of owner-occupiers are chosen at random to be buy-to-let investors.

Household financial wealth (bank balance + non-housing investments) is assigned from an empirical distribution, under the assumption that the more income a household has, the more wealthy it is.

\subsection{Assigning houses to households}
The highest quality houses are assigned to the highest income owner-occupiers. The remaining houses are assigned to buy-to-let investors, the number of houses owned is drawn from a log-normal distribution fitted to data.

If a household is assigned a home, the household is deemed to own the house outright with a tunable probability (currently 47\% of homeowners, based from ONS stats) otherwise the household is given a mortgage. In this case, the date that the house was bought is chosen at random with uniform probability over the last 25 years\footnote{25 years is the modal length of a mortgage in the UK}. The original purchase price is calculated based on a fixed inflation rate (currently set arbitrarily at 3\%) and the number of mortgage payments outstanding is reduced pro-rata. The mortgages for buy-to-let houses are arranged in a similar way.

\section{Household decision making}
\subsection{Consumption}
Households consume according to
\begin{equation}
\alpha \max(b - e^{4.07\log{(i)} - 33.1 + \epsilon},0)
\end{equation}
 where $b$ is the household's liquid wealth and $i$ is gross annual income.

\subsection{Selling houses}
At each time-step a household will decide to put each house it owns onto the market with a fixed probability (currently set at $p = \frac{1}{12*7}$). The house will be put on the market at a price, $q$ given by
\begin{equation}
\ln{q} = 0.095 + \ln{(\bar{p})} - D\ln{\left(\frac{\bar{d}+1}{\left<d\right>+1}\right)} + \epsilon
\end{equation}
where $\bar{p}$ is the average sold-price of houses of this quality, $\bar{d}$ is the average days on the market for all house qualities, $\left<d\right>$ is the prior expectation of the number of days on the market, $D$ is a tunable parameter (currently set to 0), and $\epsilon = \mathcal{N}(0,0.01^2)$.
 
If a house remains on the market from the previous time-step, its price is reduced by $5\%$. If the price drops below the amount needed to pay off the mortgage on the house, it is withdrawn from the market.

\subsection{Buying houses}
If a household decides to buy a house, it will bid an amount according to
\begin{equation}
p = \frac{\epsilon h i^g }{1 - aHPA}
\end{equation}
%% \frac{\epsilon h i^g}{\tau + c + r LTV - aHPA}
where $p$ is the desired price, $ln(\epsilon) = \mathcal{N}(0.0,0.3^2)$, $h = 4.5$, $i$ is the household's annual income, $g=1.0$, $a=0.1$, and HPA is the annualised change in the house price index. If a household can get a mortgage of value $p$, we call it a mortgage-approved household.

A mortgage-approved household will bid on the housing market immediately after it sells its home.

If a mortgage-approved household is currently renting, it will decide to bid on the housing market with probability\footnote{N.B. Boltzmann distribution reduces to this for binary decisions}
\begin{equation}
\frac{1}{1+e^{-k(C_r + C_i - C_h)}}
\end{equation}
 where $C_r$ is the current cost of renting per unit time, $C_i$ is a tunable parameter that represents the psychological cost of renting rather than owning ones own home, and $C_h = (p-d)r - pHPA$ is the immediate cost of a mortgage per unit time, where $d$ is the mortgage down-payment and $r$ is the interest rate on the mortgage per unit time.

A household may also be offered a chance to purchase a buy-to-let house if it is already a buy-to-let investor (see buy-to-let clearing). In this case, if the household can get approval on a mortgage to buy the house at the current price, it will decide to buy with a probability
\begin{equation}
p = \frac{1}{1 + e^{4.5 - 24y}}
\end{equation}
where
\begin{equation}
y = \frac{i + a}{d}
\end{equation}
where $i$ is the expected income from rental payments minus expected mortgage payments, $a$ is the expected appreciation in house price (based on current house-price appreciation) and $d$ is the down-payment on the mortgage.

\subsection{Renting}

A household will enter the rental market if it has sold its home and was not successful in buying another home, or if a current rental contract ends. A household will always bid 0.3 times its monthly income on rental payments.

A household that owns a buy-to-let house will put it on the rental market whenever a rental contract ends, or when a new buy-to-let house is bought that doesn't already have a tenant. A household will always charge rent at (1+P) times the monthly mortgage payment on the house (P is currently 0).

The length of a rental agreement is chosen randomly from 1 to 12 months with uniform probability.

\section{Bank decision making}
A bank will approve a mortgage on a house as long as it conforms to  LTV, LTI and affordability constraints. The affordability constraint ensures that a household has enough total income to pay all its mortgages. The maximum mortgage amount, then, is calculated as
\begin{equation}
q = \min\left(\frac{b}{\theta}, \frac{i\psi}{1-\theta}, b + pdi\frac{1-(1+r)^{-N}}{r}\right)
\end{equation}
where the first term is the limit due to affordability of the down-payment given the LTV constraint, $b$ is the household's bank balance, $\theta$ is the minimum loan to value haircut (i.e. 1 minus the loan to value ratio), the second term is the LTI constraint, where $\psi$ is the maximum loan to income ratio, and the last term is limit of affordability given a down-payment of all cash in the bank and a monthly payment equal to the household's disposable income, $pdi$ is a household's disposable income (including income from rent), $r$ is the monthly interest rate and $N$ is the number of monthly payments to pay off the mortgage. $\theta$ is 0.1 for first-time buyers, 0.2 for homeowners moving home and 0.4 for buy-to-let investors. $\phi$ is 0.25.

If a household applies for a mortgage, the down-payment is chosen to be a fixed fraction of the household's bank balance, or the minimum amount that is compatible with all three constraints. The interest rate is fixed at $3\%$ and the monthly payments are fixed amounts calculated to pay off the mortgage in 25 years.

\section{Housing market clearing}

Clearing is done in three phases, in the following order:

\subsection{House sales clearing}

Buyers are sorted by the value of their bid. The highest value bid is taken and matched with the highest quality house the buyer can afford (if one exists). If there is more than one house on the market at a given quality, the lowest price house is matched. If a match is made, the sale goes through immediately and the seller gets the opportunity to bid for a new house. The cycle then begins again until there are no more buyers.

\subsection{Buy-to-let clearing}
Any houses remaining on the market after house-sales-clearing are offered at random to households that already own a home. Each house is offered only once per clearing session and a household may decide not to buy the house, in which case the house remains on the market.

\subsection{Rental clearing}

Rental clearing proceeds in the same way as house-sales-clearing.

\subsection{Proposed unified house clearing}
In order to capture the effects of competition between buy-to-let investors and home-buyers on the housing market, the first two phases of the house clearing process need to be unified.

Given an equation to calculate the return on a house as a buy-to-let investment, I propose the following clearing process: All potential buyers bid at the beginning of the clearing process. Buy-to-let investors bid the difference between the current value of their portfolio and the desired value.

Buyers are sorted by the value of their bid. The highest value bid is taken and if it is a home-buyer it is matched with the highest quality house the buyer can afford (if one exists), if it is a buy-to-let investor it is matched with the highest return house he can afford. If there is more than one house on the market at a given quality/return, the lowest/highest price house is matched respectively. If a match is made, the sale goes through immediately and the seller gets the opportunity to bid for a new house. If the buyer is an investor, he gets the opportunity to bid again on the market if he has more to invest. The cycle then begins again until there are no more buyers.

%%\appendix

\end{document}
